\documentclass{report}
%\usepackage[landscape]{geometry}
\usepackage{longtable}
\usepackage{tabularx}
\usepackage[inline]{enumitem}
\usepackage{tikz}
\usetikzlibrary{calc,shapes, positioning,arrows}

\usepackage{bibleref}
\usepackage{ecjhebrew}
\newcommand{\cl}[2]{\color{black} #2 \color{#1}}

\usepackage{minted}
\usepackage{listings}
\newcommand{\mi}[1]{\lstinline{#1}}

\makeatletter
\newenvironment{python}{%
  \VerbatimEnvironment
  \minted@resetoptions
  \setkeys{minted@opt}{}
      \begin{VerbatimOut}{\jobname.pyg}}
{%
      \end{VerbatimOut}
      \minted@pygmentize{python}
      \DeleteFile{\jobname.pyg}}
\makeatother

\usepackage{multicol}

%\usepackage[usenames, dvipsnames]{color}
%\newcommand{\fixme}[1]{{#1}}

\title{Encoding options}
\author{Grietje Commelin}

\begin{document}
\maketitle

\section{Packages and stuff to be loaded}

\begin{python}
#{{{ Import all important libraries
import os, sys, time
if not '/projects/1fdd6b0b-5199-4b0e-9f7f-7759af4a5e7a/.local/lib/python3.4/site-packages/' in sys.path:
    sys.path.append('/projects/1fdd6b0b-5199-4b0e-9f7f-7759af4a5e7a/.local/lib/python3.4/site-packages/')
import collections
import random
import laf
from laf.fabric import LafFabric
import etcbc
from etcbc.preprocess import prepare
from etcbc.lib import Transcription, monad_set
from etcbc.trees import Tree
from etcbc.featuredoc import FeatureDoc
fabric = LafFabric()
tr = Transcription()

#}}}
\end{python}

\begin{python} 
#{{{ Load the API
API = fabric.load('etcbc4', '--', 'feature-doc', {
    "xmlids": {"node": False, "edge": False},
    "features": ('''
            otype
            domain
            function
            g_cons
            g_cons_utf8
            g_lex
            g_lex_utf8
            g_nme
            g_nme_utf8
            g_pfm
            g_pfm_utf8
            g_prs
            g_prs_utf8
            g_uvf
            g_uvf_utf8
            g_vbe
            g_vbe_utf8
            g_vbs
            g_vbs_utf8
            g_word
            g_word_utf8
            gn
            lex
            lex_utf8
            ls
            nme
            nu
            number
            pdp
            pfm
            prs
            ps
            rela
            sp
            st
            tab
            trailer_utf8
            txt
            typ
            uvf
            vbe
            vbs
            vs
            vt
            book
            chapter
            label
            verse
    ''','''
            mother
    '''),
    "primary": False,
    "prepare": prepare,
})
exec(fabric.localnames.format(var='fabric'))

#}}}

#{{{ Stuff necessary to load parents relationships
type_info = (
    ("word", ''),
    ("subphrase", 'U'),
    ("phrase", 'P'),
    ("phrase_atom", 'PA'),
    ("clause", 'C'),
    ("clause_atom", 'CA'),
    ("sentence", 'S'),
)
type_table = dict(t for t in type_info)
type_order = [t[0] for t in type_info]

pos_table = {
    'adjv': 'aj',
    'advb': 'av',
    'art': 'dt',
    'conj': 'cj',
    'intj': 'ij',
    'inrg': 'ir',
    'nega': 'ng',
    'subs': 'n',
    'nmpr': 'n-pr',
    'prep': 'pp',
    'prps': 'pr-ps',
    'prde': 'pr-dem',
    'prin': 'pr-int',
    'verb': 'vb',
}

ccr_info = {
    'Adju': ('r', 'Cadju'),
    'Attr': ('r', 'Cattr'),
    'Cmpl': ('r', 'Ccmpl'),
    'CoVo': ('n', 'Ccovo'),
    'Coor': ('x', 'Ccoor'),
    'Objc': ('r', 'Cobjc'),
    'PrAd': ('r', 'Cprad'),
    'PreC': ('r', 'Cprec'),
    'Resu': ('n', 'Cresu'),
    'RgRc': ('r', 'Crgrc'),
    'Spec': ('r', 'Cspec'),
    'Subj': ('r', 'Csubj'),
    'NA':   ('n', 'C'),
}

tree_types = ('sentence', 'clause', 'clause_atom', 'phrase', 'phrase_atom', 'subphrase', 'word')
(root_type, leaf_type, clause_type) = (tree_types[0], tree_types[-1], 'clause')
ccr_table = dict((c[0],c[1][1]) for c in ccr_info.items())
ccr_class = dict((c[0],c[1][0]) for c in ccr_info.items())

root_verse = {}
root_clause_atom = {}

for n in NN():
    otype = F.otype.v(n)
    if otype == "book" and F.etcbc4_sft_book.v(n) != "Genesis":
        break
    elif otype == "chapter" and int(F.etcbc4_sft_chapter.v(n)) > 5:
        break
    elif otype == 'verse': cur_verse = F.label.v(n)
    elif otype == "clause_atom":
        root_verse[n] = cur_verse
        cur_clause_atom = F.etcbc4_ft_number.v(n)
    elif otype == "word":
        root_verse[n] = cur_verse
        root_clause_atom[n] = cur_clause_atom
#        print(list(C.parent.v(n)))

tree = Tree(API, otypes = tree_types,
     clause_type=clause_type,
     ccr_feature='rela',
     pt_feature='typ',
     pos_feature='sp',
     mother_feature = 'mother',
     )
#tree.restructure_clauses(ccr_class)
results = tree.relations()
parent = results['eparent']
sisters = results['sisters']
children = results['echildren']
elder_sister = results['elder_sister']
print("Ready for processing") # Was msg()

#}}}
\end{python}

\begin{python}
#{{{ Import the helper functions
def give_me_your_words(n):
        global F
        if F.etcbc4_db_otype.v(n) == 'word':
                return [n]
        else:
            list_of_word_lists = [give_me_your_words(c) for c in children[n]]
            return [item for sublist in list_of_word_lists for item in sublist]

def get_text_of_word_list(word_list):
        global F
        return ' '.join(map(F.etcbc4_ft_g_cons_utf8.v, word_list))

def get_cons_of_word_list(word_list):
        global F
        return ' '.join(map(F.etcbc4_ft_g_cons.v, word_list))

def give_me_your_parents_up_to(node, parent_type='clause', limit=19):
    # This function is very dangerous. If you feed it something larger than a clause, stuff explodes.
        global parent
        if limit < 0:
                return []
        if not node in parent:
#                return []
                return ["not in parent"]
        p = parent[node]
        if F.etcbc4_db_otype.v(p) == parent_type:
                return [p]
        else:
                return [p] + give_me_your_parents_up_to(p, parent_type, limit-1)

def insert_dict_in_db(cursor, table, values):
    columns = ', '.join(values.keys())
    placeholders = ':'+', :'.join(values.keys())
    query = 'INSERT INTO %s (%s) VALUES (%s)' % (table, columns, placeholders)
    print(query)
    cursor.execute(query, values)

#}}}
\end{python}

\begin{python}
types = {}
DSU = False

for node in NN():
    otype = F.etcbc4_db_otype.v(node)
#    if otype == "book" and F.etcbc4_sft_book.v(node) != "Genesis":
#        break
#    elif otype == "chapter" and int(F.etcbc4_sft_chapter.v(node)) > 30:
#        break
    if otype == "verse":
        label = F.etcbc4_sft_label.v(node)
    if otype == "clause":
        DSU = (F.etcbc4_ft_txt.v(node).count("Q") > 0)    # If the text_type contains a Q, we consider the clause as Direct Speech
    elif otype == "word":
        cl_type = ""
        ph_type = ""
        ph_func = ""
        if DSU == True:
            if not (F.etcbc4_ft_ps.v(node) in ("unknown", "NA") and F.etcbc4_ft_nu.v(node) in ("unknown","NA") and F.etcbc4_ft_gn.v(node) in ("unknown","NA")):    # Potential participant
                parents = give_me_your_parents_up_to(node)
                for x in parents:
                    if F.etcbc4_db_otype.v(x) == "phrase":
                        ph_type = F.etcbc4_ft_typ.v(x)
                        ph_func = F.etcbc4_ft_function.v(x)
                    elif F.etcbc4_db_otype.v(x) == "clause":
                        cl_type = F.etcbc4_ft_typ.v(x)
                if not cl_type in types:
                    types[cl_type] = {}
                if not str((ph_func,ph_type)) in types[cl_type]:
                    types[cl_type][str((ph_func,ph_type))] = []
                if not F.etcbc4_ft_sp.v(node) in types[cl_type][str((ph_func,ph_type))]:    # Not yet saved in dictionary
                    types[cl_type][str((ph_func,ph_type))].append(str(F.etcbc4_ft_sp.v(node)))    # Add to dictionary
            if not F.etcbc4_ft_prs.v(node) in ("absent","n/a"):    # Potential participant (suffix)
                parents = give_me_your_parents_up_to(node)
                for x in parents:
                    if F.etcbc4_db_otype.v(x) == "phrase":
                        ph_type = F.etcbc4_ft_typ.v(x)
                        ph_func1 = F.etcbc4_ft_function.v(x)
                        if ph_func1 in ("ExsS","IntS","ModS","NCoS","PrcS","PreS"):
                            ph_func = "Subj"
                        elif ph_func1 in ("PreO","PtcO"): 
                            ph_func = "Objc"
                        else:
                            ph_func = "Unknown"
#                            print(F.etcbc4_ft_g_cons.v(node),"function?", F.etcbc4_ft_sp.v(node),F.etcbc4_ft_ls.v(node))
                    elif F.etcbc4_db_otype.v(x) == "clause":
                        cl_type = F.etcbc4_ft_typ.v(x)
                if not cl_type in types:
                    types[cl_type] = {}
                if not str((ph_func,ph_type)) in types[cl_type]:
                    types[cl_type][str((ph_func,ph_type))] = []
                if not "suffix" in types[cl_type][str((ph_func,ph_type))]:    # Not yet saved in dictionary
                    types[cl_type][str((ph_func,ph_type))].append("suffix")   # Add to dictionary

#}}}
\end{python}

\section{Overview \mi{clause_types}}
This is my own ordering of all \mi{clause_types} occurring in the ETCBC database, to create an overview.

\begin{itemize}
\item \mi{ clause_types["verb"]}
\begin{itemize}
\item \mi{ "Ptcp":"Participle clause"}
\item \mi{ "Voct":"Vocative clause"}
\item \mi{ "Way0":"Wayyiqtol-null clause"}
\item \mi{ "WIm0":"We-imperative-null clause"}
\item \mi{ "WQt0":"We-qatal-null clause"}
\item \mi{ "WYq0":"We-yiqtol-null clause"}
\item \mi{ "InfC":"Infinitive construct clause"}
\item \mi{ "InfA":"Infinitive absolute clause"}
\item \mi{ "ZIm0":"Zero-imperative-null clause"}
\item \mi{ "ZQt0":"Zero-qatal-null clause"}
\item \mi{ "ZYq0":"Zero-yiqtol-null clause"}
\end{itemize}
\item \mi{ clause_types["X-verb"]}
\begin{itemize}
\item \mi{ "WXIm":"We-X-imperative clause"}
\item \mi{ "WXQt":"We-X-qatal clause"}
\item \mi{ "WXYq":"We-X-yiqtol clause"}
\item \mi{ "XImp":"X-imperative clause"}
\item \mi{ "XQtl":"X-qatal clause"}
\item \mi{ "XYqt":"X-yiqtol clause"}
\end{itemize}
\item \mi{ clause_types["verb-X"]}
\begin{itemize}
\item \mi{ "WayX":"Wayyiqtol-X clause"}
\item \mi{ "WImX":"We-imperative-X clause"}
\item \mi{ "WQtX":"We-qatal-X clause"}
\item \mi{ "WYqX":"We-yiqtol-X clause"}
\item \mi{ "ZImX":"Zero-imperative-X clause"}
\item \mi{ "ZQtX":"Zero-qatal-X clause"}
\item \mi{ "ZYqX":"Zero-yiqtol-X clause"}
\end{itemize}
\item \mi{ clause_types["x-verb"]}
\begin{itemize}
\item \mi{"WxI0":"We-x-imperative-null clause"}
\item \mi{ "WxQ0":"We-x-qatal-null clause"}
\item \mi{ "WxY0":"We-x-yiqtol-null clause"}
\item \mi{ "xIm0":"x-imperative-null clause"}
\item \mi{ "xQt0":"x-qatal-null clause"}
\item \mi{ "xYq0":"x-yiqtol-null clause"}
\end{itemize}
\item \mi{ clause_types["x-verb-X"]}
\begin{itemize}
\item \mi{"WxIX":"We-x-imperative-X clause"}
\item \mi{ "WxQX":"We-x-qatal-X clause"}
\item \mi{ "WxYX":"We-x-yiqtol-X clause"}
\item \mi{ "xImX":"x-imperative-X clause"}
\item \mi{ "xQtX":"x-qatal-X clause"}
\item \mi{ "xYqX":"x-yiqtol-X clause"}
\end{itemize}
\item \mi{ clause_types["unclear"]}
\begin{itemize}
\item \mi{"CPen":"Casus pendens"}
\item \mi{ "Defc":"Defective clause atom"}
\item \mi{ "Ellp":"Ellipsis"}
\item \mi{ "MSyn":"Macrosyntactic sign"}
\item \mi{ "XPos":"Extraposition"}
\item \mi{ "Reop":"Reopening"}
\item \mi{ "Unkn":"Unknown"}
\item \mi{ "NmCl":"Nominal clause"}
\item \mi{ "AjCl":"Adjective clause"}
\end{itemize}
\end{itemize}

\section{Prominence}
\mi{Part_of_speech} occurring in direct speech words that are a potential participant (otype = \mi{"word"}, DSU = True, at least one of person/number/gender features):
Verb, subs, adjv, prde, prps, nmpr. And, or course, suffix.
Other types likely have no person/number/gender.

Proposal for prominence:
\begin{itemize}
\item 0 is reserved for `no reference at all'
\item 1 is for verbal inflection
\item 2 is for suffixes
\item 3 is for the personal pronouns
\item 4 if for the demonstrative pronouns
\item 5 is for the noun (subs), proper noun (nmpr) and adjective (adjv)
\item 6 is reserved for phrases etc, so noun+
\end{itemize}

\section{Order \mi{clause_types} to prominence}
\begin{python}
#{{{ Find (or create?) order in mess of types-dict
tuples = {}
sp = {} # list of sp for every pair of Cl_type and Ph_func
sp_coded = {}   # Same list, but coded as indicated above

for cl_type in types:
    tuples[cl_type] = sorted(types[cl_type].keys())

for cl_type in tuples:
    sp[cl_type] = {}
    sp_coded[cl_type] = {}
    for pair in tuples[cl_type]:
        ph_func = eval(pair)[0]
        if not ph_func in sp[cl_type]:
            sp[cl_type][ph_func] = []
            sp_coded[cl_type][ph_func] = []
        for z in types[cl_type][pair]:
            if not z in sp[cl_type][ph_func]:
                sp[cl_type][ph_func].append(z)
                if z == "verb":
                    if not 1 in sp_coded[cl_type][ph_func]:
                        sp_coded[cl_type][ph_func].append(1)
                elif z == "suffix":
                    if not 2 in sp_coded[cl_type][ph_func]:
                        sp_coded[cl_type][ph_func].append(2)
                elif z == "prps":
                    if not 3 in sp_coded[cl_type][ph_func]:
                        sp_coded[cl_type][ph_func].append(3)
                elif z == "prde":
                    if not 4 in sp_coded[cl_type][ph_func]:
                        sp_coded[cl_type][ph_func].append(4)
                elif z in ("subs","nmpr","adjv"):
                    if not 5 in sp_coded[cl_type][ph_func]:
                        sp_coded[cl_type][ph_func].append(5)
                else:
                    print("Problem! Found sp with type",z)
        sp_coded[cl_type][ph_func].sort()
#print(sp)

#}}}
\end{python}

\begin{python}
#{{{ Save dictionaries sp and sp_coded in gcdata.enc
import sqlite3

db = sqlite3.connect('gcdata.sqlite')
c = db.cursor()
create_table_query = open("enc.sql").read()
c.execute(create_table_query)
db.commit()

data = {}
for cl_type in sp:
    print(cl_type)
    if cl_type == "":
        data["Cl_type"] = "None"
    else:
        data["Cl_type"] = cl_type
    for ph_func in sp[cl_type]:
        if ph_func == "":
            data["Ph_func"] = "None"
        else:
            data["Ph_func"] = ph_func
        data["Enc_options"] = str(sorted(sp[cl_type][ph_func]))
        data["Enc_options_coded"] = str(sp_coded[cl_type][ph_func])
        insert_dict_in_db(c,"enc",data)

#}}}
\end{python}

\section{Idea for further processing}
For every saved participant reference in gcdata.pt we want to insert the corresponding list of \mi{sp_coded} in a column named 'Encoding options' or something like that.
Then, the index of the chosen encoding (see column PartSpeech) in this list gives the prominence of the reference given the grammatical options in the clause and phrase.

\section{New thoughts}
* Subphrases with feature ``rec'' and suffixes on nouns are similar. Together they form a category of sub-participants
* Other suffixes are either (a) in the category of PreO etc, or (b) attached to verbs or prepositions
* Other subphrases / words / phrases fulfill a slot and thus function. Find these!

\begin{python}
#{{{ Think about suffixes and their functions
DSU = False
rec_try = []    # Contains 18476 instances
subj = []       # 865 instances
objc = []       # 4269 instances
function = {}   # 9075 instances
rest = []
verbs= {}       # 35 for ptcp, 589 for ptca, 18 for infc
ET = []
notET = []
spec = []
notspec = []
spec2 = []
notspec2 = []

for node in NN():
    rec = False
    rela = ""
    otype = F.etcbc4_db_otype.v(node)
    if otype == "verse":
        label = F.etcbc4_sft_label.v(node)
    if otype == "clause":
        DSU = (F.etcbc4_ft_txt.v(node).count("Q") > 0)    # If the text_type contains a Q, we consider the clause as Direct Speech
    elif otype == "word":
        cl_type = ""
        ph_func = ""
        if DSU == True:
            if not F.etcbc4_ft_prs.v(node) in ("absent","n/a"):    # Potential participant (suffix)
                parents = give_me_your_parents_up_to(node)
                for x in parents:
                    if F.etcbc4_db_otype.v(x) == "clause":
                        cl_type = F.etcbc4_ft_typ.v(x)
                    elif F.etcbc4_db_otype.v(x) == "phrase":
                        ph_func = F.etcbc4_ft_function.v(x)

                if ph_func in ("ExsS","IntS","ModS","NCoS","PrcS","PreS"):
                    suf_func = "Subj"
                    subj.append(node)
                elif ph_func in ("PreO","PtcO"):
                    suf_func = "Objc"
                    objc.append(node)
                elif F.etcbc4_ft_sp.v(node) == "prep":
                    if not ph_func in function:
                        function[ph_func] = []
                    if ph_func == "Objc":
                        if F.etcbc4_ft_lex.v(node) == ">T":
                            ET.append((label,node))
                        else:
                            notET.append((label,node))
                        parents = give_me_your_parents_up_to(node)
                        for x in parents:
                            if F.etcbc4_db_otype.v(x) == "phrase_atom":
                                if F.etcbc4_ft_rela.v(x) == "Spec":
                                    spec.append((label,node))
                                else:
                                    notspec.append((label,node))
                    elif ph_func == "Subj":
                        parents = give_me_your_parents_up_to(node)
                        for x in parents:
                            if F.etcbc4_db_otype.v(x) == "phrase_atom":
                                if F.etcbc4_ft_rela.v(x) == "Spec":
                                    spec2.append((label,node))
                                else:
                                    notspec2.append((label,node))
                    function[ph_func].append((label,node))
                elif F.etcbc4_ft_sp.v(node) in ("subs","adjv"):
                    rec_try.append(node)
                elif F.etcbc4_ft_vt.v(node) in ("ptca","ptcp","infa","infc"):
                    if not F.etcbc4_ft_vt.v(node) in verbs:
                        verbs[F.etcbc4_ft_vt.v(node)] = []
                    verbs[F.etcbc4_ft_vt.v(node)].append(node)
                else:
                    rest.append((label,node))

#}}}
\end{python}

\section{Analyze suffixes}
function['Subj']:
Gen 6:18 (2959) and Gen 8:16 (3870): Noah --> take wife etc with you
Gen 40:8 (22045) and Gen 41:15 (22624): Object marker with suffix annotated as subject instead of negation
Gen 44:18 (25174): correct. ``Like you'' or something like that
Exo 16:32 (38022): real subject is addressee of command, but imperative annotated as subs?
Exo 19:24 (39562): You and Aaron ``with you''
Exo 29:21 (44985): His sons and the clothes of his son ``with him''

--> Is there something possible with rela-a (\mi{rela.v(phrase_atom})?) being ``Spec''?
It seems that in most cases,this Spec is present and the Subject-assignment is wrong. If \mi{rela.v(phrase_atom)} == 'NA', the assignment seems correct (this is only happening in a few cases). Cases with Spec are in list spec2, others in notspec2.
len(function['Subj']) = 59
len(spec2) = 53
len(notspec2) = 6
``Spec''-cases are comparable with function['Adju']?

function['Objc'] in most cases is a suffix on the object marker >T.
Most other cases are like ``Spec''? Listed in ET vs notET and spec vs notspec.
len(function['Objc']) =  1241
len(spec) = 32
len(notspec) = 1209
len(ET) = 1204
len(notET) = 37
Two texts are in spec and in ET: Gen 41:10 (22536) seems to be an object and Eze 16:60 (272086) seems to be a ``with you'', so not an object

7 texts are not in spec and not in ET:
    Deut 1-:21 (99550) ``with you'' / ``in your presence'' or something (pp)
    Deut 28:8 (108203) pp (blessing [over? to?] you)
    Jes 48:13 (228973) call to them
    Jer 47:2 (260254) sit in her (city). Rec
    Eze 20:35 (274290) judge with them. Might be object indeed, but in pp
    Eze 38:22 (284288) idem
    IIChr 10:11 (411950) load a yoke upon them
General conclusion: preposition is more leading than the \mi{ph_function being} ``Objc''.

1 instance of function['Voct']: addressee is a wood and all the trees ``in it''. So pp

5 instances of function['Time']: 1 in Est, 4 in Dan.
    Est 3:12 (366942) some extensive time reference, suffix unclear to me. Maybe ``in the first month, in the 30th day ``in it'' ''?
    Dan 3:6 (371706) ``In this? (suffix) hour''
    Dan 3:15 (371924) idem
    dAN 4:30 (372991) idem
    dAN 4:33 (373086) ``In this? (suffix) time''


IDEA: filter on prepositions instead of \mi{ph_func}. Might give more insight.

\begin{python}
#{{{
text = []
check = False
check2 = False

for node in NN():
    otype = F.etcbc4_db_otype.v(node)
    if otype == "book" and F.etcbc4_sft_book.v(node) != "Genesis":
        break
    elif otype == "chapter":
        if F.etcbc4_sft_chapter.v(node) in ("1","2"):
            check = False
        elif F.etcbc4_sft_chapter.v(node) == "3":
            check = True
        elif int(F.etcbc4_sft_chapter.v(node)) > 3:
            break
    elif otype == "verse":
        if F.etcbc4_sft_verse.v(node)  == "20":
            check2 = True
        else:
            check2 = False
    elif otype == "clause" and check == True and check2 == True:
        text.append(get_cons_of_word_list(give_me_your_words(node)))

#}}}

#{{{
losers = []
double = []

for (label,node) in notET:
    if (label,node) not in spec:
        losers.append((label,node))

for (label,node) in spec:
    if (label,node) in ET:
        double.append((label,node))
#}}}
\end{python}

\begin{python}
#{{{ New try to understand suffixes

DSU = False
rec_try = []    # Contains 18476 instances
subj = []       # 865 instances
objc = []       # 4269 instances
function = {}   # 9075 instances
rest = []
verbs= {}       # 35 for ptcp, 589 for ptca, 18 for infc
preps = {}

for node in NN():
    rec = False
    rela = ""
    otype = F.etcbc4_db_otype.v(node)
    if otype == "verse":
        label = F.etcbc4_sft_label.v(node)
    if otype == "clause":
        DSU = (F.etcbc4_ft_txt.v(node).count("Q") > 0)    # If the text_type contains a Q, we consider the clause as Direct Speech
    elif otype == "word":
        cl_type = ""
        ph_func = ""
        if DSU == True:
            if not F.etcbc4_ft_prs.v(node) in ("absent","n/a"):    # Potential participant (suffix)
                parents = give_me_your_parents_up_to(node)
                for x in parents:
                    if F.etcbc4_db_otype.v(x) == "clause":
                        cl_type = F.etcbc4_ft_typ.v(x)
                    elif F.etcbc4_db_otype.v(x) == "phrase":
                        ph_func = F.etcbc4_ft_function.v(x)

                if ph_func in ("ExsS","IntS","ModS","NCoS","PrcS","PreS"):
                    suf_func = "Subj"
                    subj.append(node)
                elif ph_func in ("PreO","PtcO"):
                    suf_func = "Objc"
                    objc.append(node)
                elif F.etcbc4_ft_sp.v(node) == "prep":
                    if not F.etcbc4_ft_lex.v(node) in preps:
                        preps[F.etcbc4_ft_lex.v(node)] = []
                    preps[F.etcbc4_ft_lex.v(node)].append((label,node))
                elif F.etcbc4_ft_sp.v(node) in ("subs","adjv"):
                    rec_try.append(node)
                elif F.etcbc4_ft_vt.v(node) in ("ptca","ptcp","infa","infc"):
                    if not F.etcbc4_ft_vt.v(node) in verbs:
                        verbs[F.etcbc4_ft_vt.v(node)] = []
                    verbs[F.etcbc4_ft_vt.v(node)].append(node)
                else:
                    rest.append((label,node))
#}}}
\end{python}

\section{Results of script}
Results of this script:

Suffixes can be categorized into some groups:
\begin{itemize}
\item 865 suffixes that are annotated in the ETCBC database as having a subject function. 
\item 4269 suffixes that are annotated in the ETCBC database as having a object function. 
\item 9075 suffixes on various prepositions. Sometimes prepositions can be combined into a compound preposition. In the first running of the script, these will be counted twice: once for every preposition.
\begin{itemize}
\item        \cjRL{L} 3318
\item        \cjRL{KMW} 72
\item        \cjRL{>T==} 302 ``with''
\item        \cjRL{B} 1089
\item        \cjRL{K} 13   ``as''
\item        \cjRL{<M} 332
\item        \cjRL{<L} 1165
\item        \cjRL{>L} 1033
\item        \cjRL{JT} 1  DAN 3:12 (371818) Nota accusativi, so probably not a real preposition.
\item        \cjRL{MN} 465
\item        \cjRL{>T} 1250 This is the nota objecti, which I think is not a real preposition.
\item        \cjRL{LM<N} 12
\item        \cjRL{BL<DJ} 8 GEN and JES ``without me!'' as in ``no way!''? / ``Dat is niet aan mij'' / ``without your consent'' --> seems to be an expression?. JES: ``besides me there is no God''
\item        \cjRL{LWT} 1   ESR 4:12 (379722) ``with you''
\item        \cjRL{<D} 14
\end{itemize}
\item Suffixes on nouns and adjectives, that probably have the same function as words with a \mi{Regens-rectum relation}. The ETCBC database counts 18476 of these.
\item Suffixes on verbs, without the function being annotated. There are 642 of these suffixes (35 for ptcp, 589 for ptca, 18 for infc).
\end{itemize}


\begin{python}
#{{{ Try which participant encoding types can occur with prepositions

DSU = False
prepositions = {}
candidate = ""
types = []

for node in NN():
    otype = F.etcbc4_db_otype.v(node)
    if otype == "verse":
        label = F.etcbc4_sft_label.v(node)
    if otype == "clause":
        DSU = (F.etcbc4_ft_txt.v(node).count("Q") > 0)    # If the text_type contains a Q, we consider the clause as Direct Speech
    elif otype == "phrase":
        parents = []
    elif otype == "word" and DSU == True:
        if F.etcbc4_ft_sp.v(node) == "prep":
            types = []
            candidate = label + " "
            prep = 0
            notprep = 0
            parents = (give_me_your_parents_up_to(node,'phrase_atom'))
            for x in give_me_your_words(parents[0]):# Use subphrase if available, else phrase_atom
#                if not (F.etcbc4_ft_ps.v(node) in ("unknown", "NA") and F.etcbc4_ft_nu.v(node) in ("unknown","NA") and F.etcbc4_ft_gn.v(node) in ("unknown","NA")):    # Potential participant
                candidate += " " + F.etcbc4_ft_lex.v(x)
                if not F.etcbc4_ft_sp.v(x) == "prep":
                    types.append(F.etcbc4_ft_sp.v(x))
                    notprep = 1
                elif F.etcbc4_ft_sp.v(x) == "prep":
                    prep += 1
                    if not "suffix" in types and not F.etcbc4_ft_prs.v(node) in ("absent","n/a"):    # Potential participant (suffix)
                        types.append("suffix")
                    if prep > 1:
#                        if not "prep" in types:
#                            types.append("prep")
                        if notprep == 1:
                            types.append("WARNING! DOUBLE PREP")
            if len(types) > 1 and not "WARNING! DOUBLE PREP" in types:
                types = ['>= subphrase']
            if not F.etcbc4_ft_lex.v(node) in prepositions:
                prepositions[F.etcbc4_ft_lex.v(node)] = {}
            if not str(types) in prepositions[F.etcbc4_ft_lex.v(node)]:
                prepositions[F.etcbc4_ft_lex.v(node)][str(types)] = []
            prepositions[F.etcbc4_ft_lex.v(node)][str(types)].append(candidate)

#}}}
\end{python}


\begin{python}
family = {}
mothers = set()
daughters = daughters = set()
for daughter in NN():
    for mother in C.mother.v(daughter):
        mothers.add(mother)
        daughters.add(daughter)
        family[daughter] = mother

\end{python}

Alternative: Ci.mother.v(mother) gives list of daughters (?)
So, C and Ci are forward and backward.
If we have a mother, we can ask: \\
  for x in Ci.mother.v(mother): \\
    print(x)

\begin{python}
#{{{ Try to understand prepositions better

DSU = False
prepositions = {}
candidate = ""
types = []
start = False
done = False

for node in NN():
    otype = F.etcbc4_db_otype.v(node)
    if otype == "verse":
        label = F.etcbc4_sft_label.v(node)
    if otype == "clause":
        DSU = (F.etcbc4_ft_txt.v(node).count("Q") > 0)    # If the text_type contains a Q, we consider the clause as Direct Speech
    elif otype == "phrase":
        parents = []
    elif otype == "word" and DSU == True:
        if F.etcbc4_ft_sp.v(node) == "prep":
            types = []
            candidate = label + " "
            prep = 0
            notprep = 0
            start = False
            done = False
            if not F.etcbc4_ft_prs.v(node) in ("absent","n/a"):    # Potential participant (suffix)
                types.append("suffix")
                done = True
            parents = (give_me_your_parents_up_to(node,'phrase_atom'))
            for x in give_me_your_words(parents[0]):# Use subphrase if available, else phrase_atom
                if done == False:
                    if x == node:   # Make sure that we start only after the preposition
                        start = True
                    if start == True:
                        candidate += " " + F.etcbc4_ft_g_cons.v(x)
                        if not (F.etcbc4_ft_sp.v(x) == "prep" and prep == 0):
                            types.append(F.etcbc4_ft_sp.v(x)) 
                        if not F.etcbc4_ft_prs.v(x) in ("absent","n/a"):
                            types.append("suffix")
                            done = True
                        if not F.etcbc4_ft_sp.v(x) == "prep":
                            notprep = 1
                        elif F.etcbc4_ft_sp.v(x) == "prep":
                            prep += 1
                            if prep > 1 and notprep == 1:
                                types.append("WARNING! DOUBLE PREP")
                        if F.etcbc4_ft_sp.v(x) in ("art","prep","conj","intj","nega"):
                            done = False
                        elif x in mothers:
                            for y in Ci.mother.v(x):
                                types.append(F.etcbc4_ft_sp.v(y))
                            done = True
                        else:
                            done = True
            if len(types) > 2 and not "WARNING! DOUBLE PREP" in types:
                types = ['>= subphrase']
            if not F.etcbc4_ft_lex.v(node) in prepositions:
                prepositions[F.etcbc4_ft_lex.v(node)] = {}
            if not str(types) in prepositions[F.etcbc4_ft_lex.v(node)]:
                prepositions[F.etcbc4_ft_lex.v(node)][str(types)] = []
            prepositions[F.etcbc4_ft_lex.v(node)][str(types)].append(candidate)

#}}}
\end{python}

\subsection{Encoding options with prepositions}
The Old Testament has many prepositions, usually followed by participant references. The option of encoding a participant by (verbal) inflection obviously is not available for prepositions, since these do not have inflection. They just serve as an introduction to a participant reference, and do not form part of the reference themselves. Except for inflection (and zero encoding), all options are available for prepositions, as will be shown with examples below.

\begin{itemize}
\item With suffix: \bibleverse{Gen}(4:10) -- \cjRL{QWL DMY >XYK Y<QJM \cl{red}{>LJ} MN H>DMH} -- ``Your brother's blood cries to me from the ground.''
\item With a personal pronoun: \bibleverse{Ex}(36:1) -- \cjRL{NTN JHWH XKMH WTBWNH \cl{red}{BHMH}} -- ``... JHWH has put wisdom and understanding in them... [GC]''
\item With a demonstrative pronoun: \bibleverse{Gen}(2:23) -- \cjRL{\cl{red}{LZ>T} YQR> >CH} -- ``She will be called `woman'... ''
\item With an interrogative pronoun:  \bibleverse{Gen}(15:8) -- \cjRL{>DNJ JHWH B MH >D< KJ >JRCNH} -- ``Lord Yahweh, how will I know that I will inherit it?''
\item With a verb (participle): \bibleverse{Gen}(1:14) -- \cjRL{JHJ M>RT BRQJ< HCMJM \cl{red}{LHBDJL} BJN HJWM WBJN HLJLH} -- ``Let there be lights in the expanse of the sky to divide the day from the night''
\item With article and noun: \bibleverse{Gen}(4:10) -- \cjRL{QWL DMY >XYK Y<QJM >LY \cl{red}{MN H>DMH}} -- ``Your brother's blood cries to me from the ground.''
\item With a noun with suffix:
\item With a personal noun:
\item With an adjective:
\end{itemize}

\subsubsection{>T}
A special category form the nota objecti \cjRL{>T} and the nota accusativi \cjRL{JT} (occurring only once, in \bibleverse{Dan}(3:12)). These introduce an object slot, and have the same options as other prepositions, except for the personal pronoun, which does not occur anywhere in the Old Testament.
\begin{itemize}
\item With suffix: \bibleverse{Gen}(7:1) -- \cjRL{KJ \cl{red}{>TK} R>JTJ YDJQ} -- ``I have seen your righteousness before me in this generation.''
\item With a demonstrative pronoun: \bibleverse{Gen}(29:33) -- \cjRL{KJ CM< JHWH KJ FNW>H >NKJ W JTN LJ GM \cl{red}{>T ZH}} -- ``Because Yahweh has heard that I am hated, he has therefore given me this son also.''
\item With an interrogative pronoun: \bibleverse{IIKings}(19:22) -- \cjRL{\cl{red}{>T MJ} XRPT} --  ``Whom have you defied?''
\item With a verb (participle): \bibleverse{Gen}(23:6) -- \cjRL{BMBXR } --
\item With article and noun: \bibleverse{Gen}(4:12) \cjRL{T<BD >T H>DMH} ``When you till the ground...''
\item With a noun with suffix: \bibleverse{Gen}(22:2) \cjRL{QX N> -- \cl{red}{>T BNK} >T JXJDK >CR >HBT >T JYXQ} -- ``Now take your son, your only [son], whom you love, Isaac [GC]''
\item With a personal noun: \bibleverse{Gen}(22:2) -- \cjRL{QX N> >T BNK >T JXJDK >CR >HBT \cl{red}{>T JYXQ}.} -- ``Now take your son, your only [son], whom you love, Isaac [GC]''
\item With an adjective: \bibleverse{Gen}(22:2) -- \cjRL{QX N> >T BNK \cl{red}{>T JXJDK} >CR >HBT >T JYXQ} -- ``Now take your son, your only [son], whom you love, Isaac [GC]''
\end{itemize}
These options can, of course, be extended by subphrases or elaborate descriptions of the object, thus creating an extensive participant reference. The reference to \bibleverse{Gen}(22:2) can serve as an example here.

\\end{python}

\\begin{python}
\section{tests}
\begin{python}
# Find words of a text
text = ""
for node in NN():
    otype = F.etcbc4_db_otype.v(node)
    if otype == "book" and F.etcbc4_sft_book.v(node) != "Genesis":
        break
    elif otype == "chapter":
        chapter = F.etcbc4_sft_chapter.v(node)
    elif otype == "verse":
        verse = F.etcbc4_sft_verse.v(node)
    elif otype == "subphrase" and chapter == "6" and verse == "14":
        if F.etcbc4_ft_rela.v(node) == "rec":
            REG = ""
            for x in C.mother.v(node):
                REG = x
                print(x)
            print("REG: ",F.etcbc4_ft_g_cons.v(REG)," rectum: ", get_cons_of_word_list(give_me_your_words(node)))
            print(node)
    elif otype == "word" and chapter == "10" and verse == "09":
        text += " " + F.etcbc4_ft_g_cons.v(node)
#print(text)

\end{python}

\begin{python}

for node in NN():
    otype = F.etcbc4_db_otype.v(node)
#    if otype == "book" and not F.etcbc4_sft_book.v(node) == "Genesis":
#        break
    if otype == "verse":
        label = F.etcbc4_sft_label.v(node)
#    elif otype == "clause":
#        DSU = (F.etcbc4_ft_txt.v(node).count("Q") > 0)    # If the text_type contains a Q, we consider the clause as Direct Speech
    elif otype in ("clause","phrase","subphrase"):
        if ">JC >XJW" in get_cons_of_word_list(give_me_your_words(node)):
            print(label)

\end{python}

\end{document}
%sagemathcloud={"latex_command":"pdflatex -synctex=1 -shell-escape -interact=nonstopmode encoding_options_version2.0.tex"}
