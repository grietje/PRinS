\documentclass{article}
\usepackage[landscape]{geometry}
\usepackage{longtable}
\usepackage{tabularx}
\usepackage[inline]{enumitem}
\usepackage{tikz}
\usetikzlibrary{calc,shapes, positioning,arrows}

\usepackage{bibleref}

\usepackage{minted}
\usepackage{listings}
\newcommand{\mi}[1]{\lstinline{#1}}

\usepackage{multicol}

%\usepackage[usenames, dvipsnames]{color}
%\newcommand{\fixme}[1]{{#1}}

\title{Work flow with gcdata.sqlite and several Python scripts}
\author{Grietje Commelin}

\begin{document}
\maketitle
\lstset{language=python,basicstyle=\ttfamily}

\begin{multicols}{2}
In this task I want to extract information from the ETCBC database + ask for user-input about DSU's and QF's + put it into an SQLite-table.
A DSU is a Direct\_Speech\_Unit, and a QF is a Quotative\_Frame.

\textbf{WARNING:} DSU's are found on the basis of \mi{text\_type}s of clause\_atoms as stored in the ETCBC database. However, testing shows that these are not 100\% correct and will need improvement.

\textbf{WARNING2:} This task assumes that a DSU ends when the embedding (number of \mi{Q}'s in the \mi{text\_type} of a \mi{clause\_atom}) decreases.

However, it might very well be possible that various DSU's follow each other on the same level of embedding without intervening lower-\mi{Q} \mi{clause\_atom}s.
At the moment these cases will not be detected by this script!
\end{multicols}

\begin{minted}{python}
#{{{ Import all important libraries
import os, sys, time
if not '/projects/1fdd6b0b-5199-4b0e-9f7f-7759af4a5e7a/.local/lib/python3.4/site-packages/' in sys.path:
    sys.path.append('/projects/1fdd6b0b-5199-4b0e-9f7f-7759af4a5e7a/.local/lib/python3.4/site-packages/')
import collections
import random
import laf
from laf.fabric import LafFabric
import etcbc
from etcbc.preprocess import prepare
from etcbc.lib import Transcription, monad_set
from etcbc.trees import Tree
from etcbc.featuredoc import FeatureDoc
fabric = LafFabric()
tr = Transcription()

print("It works so far." )
#}}}
\end{minted}

\begin{minted}{python}
#{{{ Load the API
API = fabric.load('etcbc4', '--', 'feature-doc', {
    "xmlids": {"node": False, "edge": False},
    "features": ('''
            otype
            domain
            function
            g_cons
            g_cons_utf8
            g_lex
            g_lex_utf8
            g_nme
            g_nme_utf8
            g_pfm
            g_pfm_utf8
            g_prs
            g_prs_utf8
            g_uvf
            g_uvf_utf8
            g_vbe
            g_vbe_utf8
            g_vbs
            g_vbs_utf8
            g_word
            g_word_utf8
            gn
            lex
            lex_utf8
            ls
            nme
            nu
            number
            pdp
            pfm
            prs
            ps
            rela
            sp
            st
            tab
            trailer_utf8
            txt
            typ
            uvf
            vbe
            vbs
            vs
            vt
            book
            chapter
            label
            verse
    ''',""),
    "primary": False,
})
exec(fabric.localnames.format(var='fabric'))
#}}}
\end{minted}

\begin{minted}{python}
#{{{ Stuff necessary to load parents relationships
type_info = (
    ("word", ''),
    ("subphrase", 'U'),
    ("phrase", 'P'),
    ("clause", 'C'),
    ("sentence", 'S'),
)
type_table = dict(t for t in type_info)
type_order = [t[0] for t in type_info]

pos_table = {
    'adjv': 'aj',
    'advb': 'av',
    'art': 'dt',
    'conj': 'cj',
    'intj': 'ij',
    'inrg': 'ir',
    'nega': 'ng',
    'subs': 'n',
    'nmpr': 'n-pr',
    'prep': 'pp',
    'prps': 'pr-ps',
    'prde': 'pr-dem',
    'prin': 'pr-int',
    'verb': 'vb',
}

ccr_info = {
    'Adju': ('r', 'Cadju'),
    'Attr': ('r', 'Cattr'),
    'Cmpl': ('r', 'Ccmpl'),
    'CoVo': ('n', 'Ccovo'),
    'Coor': ('x', 'Ccoor'),
    'Objc': ('r', 'Cobjc'),
    'PrAd': ('r', 'Cprad'),
    'PreC': ('r', 'Cprec'),
    'Resu': ('n', 'Cresu'),
    'RgRc': ('r', 'Crgrc'),
    'Spec': ('r', 'Cspec'),
    'Subj': ('r', 'Csubj'),
    'NA':   ('n', 'C'),
}

tree_types = ('sentence', 'clause', 'clause_atom', 'phrase', 'subphrase', 'word')
(root_type, leaf_type, clause_type) = (tree_types[0], tree_types[-1], 'clause')
ccr_table = dict((c[0],c[1][1]) for c in ccr_info.items())
ccr_class = dict((c[0],c[1][0]) for c in ccr_info.items())

root_verse = {}
root_clause_atom = {}

for n in NN():
    otype = F.otype.v(n)
    if otype == "book" and F.etcbc4_sft_book.v(n) != "Genesis":
        break
    elif otype == "chapter" and int(F.etcbc4_sft_chapter.v(n)) > 5:
        break
    elif otype == 'verse': cur_verse = F.label.v(n)
    elif otype == "clause_atom":
        root_verse[n] = cur_verse
        cur_clause_atom = F.etcbc4_ft_number.v(n)
    elif otype == "word":
        root_verse[n] = cur_verse
        root_clause_atom[n] = cur_clause_atom
#        print(list(C.parent.v(n)))

tree = Tree(API, otypes = tree_types,
     clause_type=clause_type,
     ccr_feature='rela',
     pt_feature='typ',
     pos_feature='sp',
     mother_feature = 'mother',
     )
#tree.restructure_clauses(ccr_class)
results = tree.relations()
parent = results['eparent']
sisters = results['sisters']
children = results['echildren']
elder_sister = results['elder_sister']
print("Ready for processing") # Was msg()
#}}}
\end{minted}

\begin{minted}{python}
#{{{ Import the helper functions
def give_me_your_words(n):
        global F
        if F.etcbc4_db_otype.v(n) == 'word':
                return [n]
        else:
            list_of_word_lists = [give_me_your_words(c) for c in children[n]]
            return [item for sublist in list_of_word_lists for item in sublist]

def get_text_of_word_list(word_list):
        global F
        return ' '.join(map(F.etcbc4_ft_g_cons_utf8.v, word_list))

def get_cons_of_word_list(word_list):
        global F
        return ' '.join(map(F.etcbc4_ft_g_cons.v, word_list))

def give_me_your_parents_up_to(node, parent_type='clause', limit=19):
    # This function is very dangerous. If you feed it something larger than a clause, stuff explodes.
        global parent
        if limit < 0:
                return []
        if not n in parent:
                return ["not in parent"]
        p = parent[node]
        if F.etcbc4_db_otype.v(p) == parent_type:
                return [p]
        else:
                return [p] + give_me_your_parents_up_to(p, parent_type, limit-1)

def insert_dict_in_db(cursor, table, values):
        columns = ', '.join(values.keys())
        placeholders = ':'+', :'.join(values.keys())
        query = 'INSERT INTO %s (%s) VALUES (%s)' % (table, columns, placeholders)
#        print(query)
        cursor.execute(query, values)
#}}}
\end{minted}

\begin{minted}{python}
#{{{ Create an SQLite-table with a row for every QF + DSU, and columns for useful information about these DSU's.
import sqlite3
db = sqlite3.connect('gcdata.sqlite')
c = db.cursor()

create_table_query = open("gcdsu.sql").read()

c.execute(create_table_query)

db.commit()
#}}}
\end{minted}

\begin{multicols}{2}
Now we want to determine which \mi{clause_atom_number}s constitute QF's, and gather \mi{user_input} about them.
\begin{itemize}
 \item Show 4 clauses before DSU + first 4 clauses of DSU itself, all with \mi{clause_atom_number} and \mi{g_consonant}s.
 \item Ask for block of \mi{clause_atom}s before DSU:
 \begin{itemize}
    \item Which \mi{clause_atom}s (if any) are QF
    \item What "MySp" is. Can be any name
    \item What "QFSp" is. Can be anything, or \mi{NULL}
    \item What "QFSptype" is. Options: \mi{Noun}, \mi{Noun_phrase}, \mi{Name}, \mi{Name_phrase}, \mi{Pers_pronoun}, \mi{Suffix}, \mi{Verb}
    \item What "MyAd" is. Can be any name
    \item What "QFAd" is. Can be anything, or \mi{NULL}
    \item What "QFAdtype" is. Options: \mi{Noun}, \mi{Noun_phrase}, \mi{Name}, \mi{Name_phrase}, \mi{Pers_pronoun}, \mi{Suffix}, \mi{Verb}
    \item What "QFAdprep" is
    \item What "QFplus" is. Can be anything, or \mi{NULL}
    \item What "QFplustype" is. Can be something like \mi{Time} or \mi{Space}, or \mi{NULL}
  \end{itemize}
\end{itemize}
\end{multicols}

\begin{minted}{python}
#{{{ Verbs
def verbs(QFClANrs):
    QFverb,QFverbspeech,QFLMR = [],[],"No"
    for x in QFClANrs:
        for y in get_text_of_word_list(give_me_your_words(x)):
            if F.etcbc4_ft_sp.v(y) == "verb":
                QFverb.append(y)
                if F.etcbc4_ft_ls.v(y) == "quot":
                    QFverbspeech.append(y)
                    if F.etcbc4_ft_lex.v(y) == ">MR[" and F.etcbc4_ft_vt.v(y) == "infc":
                        QFLMR = "Yes"
    return(','.join(str(x) for x in QFverb),','.join(str(x) for x in QFverbspeech),QFLMR)

#}}}
\end{minted}

\begin{minted}{python}
#{{{ Process_DSU
def process_DSU(ClANrs, potential_QF, place,prev_context,context,level):
    DSUdata = {}
    print(place)
    print("Potential QF:")  # Show the clause_atom_numbers of potential_QF with their text
    for x in potential_QF:
        print('{0:<8} {1:>35} {2} {3:<35}'.format(x, get_text_of_word_list(give_me_your_words(x))," | ", get_cons_of_word_list(give_me_your_words(x))))
    print("Start of DSU:")  # Show the first clause_atom_numbers and text of the DSU
    for y in ClANrs[:3]:
        print('{0:<8} {1:>35} {2} {3:<35}'.format(y, get_text_of_word_list(give_me_your_words(y))," | ",get_cons_of_word_list(give_me_your_words(y))))
    print("-"*20)
    DSUdata["a"] = place
    DSUdata["Level"] = level
#    DSUdata["ClANrs"] = ','.join(str(x) for x in ClANrs)
    DSUdata["ClANrs"] = repr(ClANrs)
    DSUdata["Con1"] = prev_context
    DSUdata["Con2"] = context
    DSUdata["QFClANrs"] = input("Clause_atoms that are indeed QF (type comma-separated numbers): ") or ""
    DSUdata["QFverb"],DSUdata["QFverbspeech"],DSUdata["QFLMR"] = verbs(DSUdata["QFClANrs"])
    DSUdata["QFSp"] = input("QFSp: ") or None
    DSUdata["QFSptype"] = input("QFSptype (1=Noun, 2=Noun_phrase, 3=Name, 4=Name_phrase, 5=Pers_pronoun, 6=Other_pronoun, 7=Suffix, 8=Verb): ") or None
    DSUdata["MySp"] = repr(input("MySp: ").split(",")) or None
    DSUdata["QFAd"] = input("QFAd: ") or None
    DSUdata["QFAdtype"] = input("QFAdtype (1=Noun, 2=Noun_phrase, 3=Name, 4=Name_phrase, 5=Pers_pronoun, 6=Other_pronoun, 7=Suffix, 8=Verb): ") or None
    DSUdata["QFAdprep"] = input("QFAdprep: ") or None
    DSUdata["MyAd"] = repr(input("MyAd: ").split(",")) or None
    DSUdata["QFplus"] = input("QFplus: ") or None
    DSUdata["QFplustype"] = input("QFplustype: ") or None
    DSUdata["Start_node"] = ClANrs[0]
    DSUdata["DSUtag"] = str(DSUdata["Start_node"]) + str(DSUdata["Level"])
    if DSUdata["QFClANrs"] != "":
        DSUdata["QF"] = "Yes"
    elif DSUdata["QFClANrs"] == "":
        DSUdata["QF"] = "No"
    insert_dict_in_db(c, "gcdsu", DSUdata)

#}}}
\end{minted}

\begin{minted}{python}
#{{{ Main function
def main_loop():
    ClANrs =       [[],[],[],[],[],[],[],[]]
    potential_QF = [[],[],[],[],[],[],[],[]]
    pot_QF = []
    context = ""
    prev_context = ["","","","","","","",""]

    place = None
    places =       ["","","","","","","",""]
    level = 0
    completed = False
    check_node = ""

    for node in NN():
        otype = F.etcbc4_db_otype.v(node)

        if otype == "book" and F.etcbc4_sft_book.v(node) != "Genesis":
            break
        elif otype == "chapter":
            if int(F.etcbc4_sft_chapter.v(node)) < 2:    # Prevent re-doing already completed chapters
                completed = True
            elif int(F.etcbc4_sft_chapter.v(node)) > 3:    # Set a limit on the amount of data dealt with this time
                break
            else:
                completed = False

        elif otype == "verse" and completed == False:
            place = F.etcbc4_sft_label.v(node)
        elif otype == "clause" and completed == False:
            prev_level = level
            level = F.etcbc4_ft_txt.v(node).count("Q")      # Determine level by parsing text_type data
            previous_context = context
            context = F.etcbc4_ft_txt.v(node)

        elif otype == "clause_atom" and completed == False:
            pot_QF.append(node)     # Update pot_QF
            if len(pot_QF) > 5:
                del pot_QF[0]       # Keep length of pot_QF reasonable
            if level > 0:           # At least one DSU under construction
                for i in range(1,(level+1)):
                    ClANrs[level].append(node)   # For all DSU's under construction, add current ClANr to list of ClANrs
            if prev_level < level:                  # Start of new DSU
                places[level] = place               # Store starting place of new DSU
                potential_QF[level] = pot_QF[:-1]   # Store potential QF of new DSU
                prev_context[level] = previous_context    # Store context (preceding text_type)

            elif prev_level > level:                # End of one or more DSU's
                for j in range((level+1),(prev_level+1)): # Process the DSU's that end here
                    check_node = str(ClANrs[j][0])
                    c.execute("SELECT EXISTS (SELECT * FROM gcdsu WHERE Start_node = " + check_node + ");") # Check whether DSU is not yet saved in database
                    result = c.fetchone()
                    if result == (0,):
                        process_DSU(ClANrs[j], potential_QF[j],places[j],prev_context[j],context,j)
                    ClANrs[j], potential_QF[j], places[j], prev_context[j] = [],[],"",""    # Delete all stored data

#}}}
\end{minted}

\begin{minted}{python}
#{{{ Execute main loop
main_loop()
#}}}

#{{{ Wrap up the database connection
db.commit()
c.close()
#}}}
\end{minted}

\end{document}

%sagemathcloud={"latex_command":"pdflatex -synctex=1 -shell-escape -interact=nonstopmode DSU+QF_version3.1.tex"}