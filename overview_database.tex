\documentclass{article}
\usepackage[landscape]{geometry}
\usepackage{longtable}

\usepackage[usenames, dvipsnames]{color}
\newcommand{\fixme}[1]{\color{red} #1}

\title{Overview of database structure GCdata}
\author{Grietje Commelin}

\begin{document}
\maketitle

\section{GCdata.gcdsu}


\begin{longtable}{|l|p{.7\textwidth}|l|}
\hline
\emph{Column} & \emph{Description} & \emph{How?}\\ \hline
ID          & Automatically created key & DSU+QF \\ \hline
Level       & Level of embedding (number of Q's in text\_type (txt)) & DSU+QF \\ \hline
a           & Place in corpus where DSU proper starts: Book, chapter, verse & DSU+QF \\ \hline
Start\_node & ETCBC node\_ID of first clause\_atom & DSU+QF \\ \hline
ClANrs      & List of clause\_atom ID's in DSU, as indicated by ETCBC & DSU+QF \\ \hline
QFClANrs    & List of clause\_atom ID's in QF, as indicated by ETCBC & DSU+QF input \\ \hline
Con1        & Textual context (txt) before the slot of the Quotative Frame (QF): Discursive, Narrative, Quotation, or Unknown (or a combination of these) & DSU+QF \\ \hline
Con2        & Textual context (txt) after the DSU & DSU+QF \\ \hline
QF          & Presence or absence of QF (Yes or No) & DSU+QF \\ \hline
QFverb      & Verbs occurring in QF & DSU+QF \\ \hline
QFverbspeech & Verbs with F.etcbc4\_ft\_ls.v(node) == "quot" & DSU+QF \\ \hline
QFLMR       & Presence or absence of LMR in QF & DSU+QF \\ \hline
QFSp        & Mentioning of speaker in QF: text of mentioning & DSU+QF input \\ \hline
QFSptype    & Mentioning of speaker in QF: type of mentioning (Noun, Noun\_phrase, Name, Name\_phrase, Pers\_pronoun, Other\_pronoun, Suffix, Verb) & DSU+QF input \\ \hline
MySp        & My impression of the speaker of the DSU & DSU+QF input \\ \hline
QFAd        & Mentioning of addressee in QF: text of mentioning & DSU+QF input \\ \hline
QFAdtype    & Mentioning of addressee in QF: type of mentioning (Noun, Noun\_phrase, Name, Name\_phrase, Pers\_pronoun, Other\_pronoun, Suffix, Verb) & DSU+QF input \\ \hline
MyAd        & My impression of the addressee of the DSU & DSU+QF input \\ \hline
QFAdprep    & Preposition used to introduce addressee in QF & DSU+QF input \\ \hline
QFplus      & Extra information in QF: text of mentioning & DSU+QF input \\ \hline
QFplustype  & Extra information in QF: type of mentioning (something like Time or Space) & DSU+QF input \\ \hline
Part        & Participants involved in this DSU & Part-2.py \\ \hline
Clos        & Open or closed conversation & \\ \hline
DSUtag      & Unique tag, composed of Start\_node and Level & DSU+QF \\ \hline
MaxQFSp     & Optional field if QF does contain a more extensive reference than the QFSp & ?\\ \hline
MaxQFAd     & Optional field if QF does contain a more extensive reference than the QFAd & ? \\ \hline
MillerQF    & List of clause\_atom ID's in QF according to Miller, thus including metapragmatic verbs (if more extensive than QFClANrs) & ?\\ \hline
MillerSp    & Text of mentioning of the speaker according to Miller's theory & ? \\ \hline
MillerSptype & Type of mentioning of the speaker according to Miller's theory & ? \\ \hline
MillerAd    & Text of mentioning of the addressee according to Miller's theory & ? \\ \hline
MillerAdtype & Type of mentioning of the addressee according to Miller's theory & ? \\ \hline
MillerAdprep & Preposition used to introduce addressee according to Miller's theory & ? \\ \hline
LongQF      & List of clause\_atom ID's in QF according to Longacre, thus including metapgramatic verbs, motion verbs and psychological verbs (if more extensive than QFClANrs) & ? \\
LongSp      & Text of mentioning of the speaker according to Longacre's theory & ? \\ \hline
LongSptype  & Type of mentioning of the speaker according to Longacre's theory & ? \\ \hline
LongAd      & Text of mentioning of the addressee according to Longacre's theory & ? \\ \hline
LongAdtype  & Type of mentioning of the addressee according to Longacre's theory & ? \\ \hline
LongAdprep  & Preposition used to introduce addressee according to Longacre's theory & ? \\ \hline

\end{longtable}

\section{GCdata.pt}

\begin{longtable}{|l|p{.7\textwidth}|l|}
\hline
\emph{Column} & \emph{Description} & \emph{How?}\\ \hline
\multicolumn{3}{|l|}{\bf{Place where reference is found}} \\ \hline
ID & Automatically created ID (of SQLite entry) & Part-1.py \\ \hline
a & Place in corpus where participant reference is found (book, chapter, verse) & Part-1.py \\ \hline
Node & Node ID in ETCBC data & Part-1.py \\ \hline
Nodetype & Type of node: 1=part\_suffix, 2=part\_word, 3=part\_(sub)phrase, 4=part\_clause, 5=part\_sentence & Part-1.py input\\ \hline
ClANr & Clause atom ID & Part-1.py \\ \hline
PhANr & Phrase atom ID & Part-1.py \\ \hline
WNr & Word ID & Part-1.py \\ \hline
Level & Levels of embedding (of whole DSU) as computed by text\_type and stored in GCdata.db & Part-2.py\\ \hline
Emb & Levels of embedding as indicated by ETCBC & \fixme{F.etcbc4\_ft\_tab.v(node)} \\ \hline
DSU & ID of the DSU to which reference belongs & Part-2.py \\ \hline

\multicolumn{3}{|l|}{\bf{Place where first found}} \\ \hline

a1 & Place in corpus where participant is found for the first time & Part-2.py \\ \hline
Node1 & Node ID (in ETCBC data) of first reference to this participant & Part-2.py \\ \hline
Nodetype1 & Node type of first reference: 1=part\_suffix, 2=part\_word, 3=part\_phrase, 4=part\_clause, 5=part\_sentence & Part-2.py\\ \hline
ClANr1 & Clause\_atom ID of first reference to this participant & Part-2.py\\ \hline
PhANr1 & Phrase\_atom ID & Part-2.py\\ \hline
WNr1 & Word number (within book) & Part-2.py\\ \hline
DSU1 & ID of the DSU to which first reference belongs & Part-2.py\\ \hline
Name & Identifying name / label for this participant (referent) & Part-1.py \\ \hline

\multicolumn{3}{|l|}{\bf{Previous participant reference}} \\ \hline

Preva & Place in corpus where previous participant reference is found (book, chapter, verse) & Part-2.py \\ \hline
PrevNode & Node ID of previous reference in ETCBC data & Part-2.py \\ \hline
PrevNodetype & Type of node: 1=part\_suffix, 2=part\_word, 3=part\_phrase, 4=part\_clause, 5=part\_sentence & Part-2.py \\ \hline
PrevClANr & Clause atom ID of previous reference & Part-2.py \\ \hline
PrevPhANr & Phrase atom ID of previous reference & Part-2.py \\ \hline
PrevWNr & Word number (within book) of previous reference & Part-2.py \\ \hline
PrevCon & Context of previous participant reference in text &  Part-2.py\\ \hline
PrevLevel & Levels of embedding as computed by text\_type and stored in GCdata.db & Part-2.py \\ \hline
PrevEmb & Levels of embedding as indicated by ETCBC & Part-2.py \\ \hline
PrevDSU & ID of the DSU to which previous reference belongs & Part-2.py \\ \hline
PrevDes & Surface consonants of previous participant reference in text & Part-2.py \\ \hline
PrevDestype & Type of designation of previous reference in text & Part-2.py \\ \hline
PrevRole & Role or grammatical function of previous reference in text & Part-2.py \\ \hline
PrevP & Person of previous reference & Part-2.py \\ \hline
PrevN & Number of previous reference & Part-2.py \\ \hline
PrevG & Gender of previous reference & Part-2.py \\ \hline
PrevLexset & Lexical set of previous reference & Part-2.py \\ \hline
PrevPartSpeech & Part of speech of previous reference & Part-2.py \\ \hline
PrevName & Identifying name / label of participant in previous reference & Part-2.py \\ \hline

\multicolumn{3}{|l|}{\bf{Previous reference to same participant}} \\ \hline
Refa & Place in corpus where previous reference to this participant is found (book, chapter, verse) & Part-2.py \\ \hline
RefNode & Node ID of previous reference to this participant in ETCBC data & Part-2.py \\ \hline
RefNodetype & Type of node: 1=part\_suffix, 2=part\_word, 3=part\_phrase, 4=part\_clause, 5=part\_sentence & Part-2.py \\ \hline
RefClANr & Clause atom ID of previous reference to same participant & Part-2.py \\ \hline
RefPhANr & Phrase atom ID of previous reference to same participant & Part-2.py \\ \hline
RefWNr & Word number (within book) of previous reference to same participant & Part-2.py \\ \hline
RefCon & Context of previous reference to same participant & Part-2.py \\ \hline
RefLevel & Levels of embedding as computed by text\_type and stored in GCdata.db & Part-2.py \\ \hline
RefEmb & Levels of embedding as indicated by ETCBC & Part-2.py \\ \hline
RefDSU & ID of the DSU to which previous reference to same participant belongs & Part-2.py \\ \hline
RefDes & Surface consonants of previous reference to same participant & Part-2.py \\ \hline
RefDestype & Type of designation by which this participant is mentioned in previous reference & Part-2.py \\ \hline
RefRole & Role or grammatical function of previous reference to same participant & Part-2.py \\ \hline
RefP & Person of previous reference to same participant & Part-2.py \\ \hline
RefN & Number of previous reference to same participant & Part-2.py \\ \hline
RefG & Gender of previous reference to same participant & Part-2.py \\ \hline
RefLexset & Lexical set of previous reference to same participant & Part-2.py \\ \hline
RefPartSpeech & Part of speech of previous reference to same participant & Part-2.py \\ \hline
RefInt & Number of different intervening participants between current and Ref & \\ \hline

\multicolumn{3}{|l|}{\bf{Current reference}} \\ \hline

Con & Context of current reference & \\ \hline
Des & Surface consonants of designation & Part-1.py \\ \hline
Destype & Type of designation by which this participant is mentioned (Noun, Noun\_phrase, Name, Name\_phrase, Pers\_pronoun, Other\_pronoun,  Suffix, Verb) & Part-1.py input \\ \hline
Role & Role or grammatical function within the sentence (??) == F.etcbc4\_ft\_function.v(node) & \\ \hline
P & Person of current reference & Part-1.py \\ \hline
N & Number of current reference & Part-1.py \\ \hline
G & Gender of current reference & Part-1.py \\ \hline
PartSpeech & Part of speech of current reference (article, verb, noun, proper noun, adverb, preposition, conjunction, pers. pronoun, demons. pron, interr. pronoun, interjection, negative, interrogative, adjective) & Part-1.py \\ \hline
Lexset & Lexical set of current reference(distributive noun, copulative noun, potential adverb, anaphoric adverb, potential preposition, conjunctive adverb, ordinal, copulative verb, noun of multitude, focus particle, interrogative particle, gentilic, quotation verb, cardinal)\footnote{Be aware: might be wrong in case of higher level reference type (e.g. part\_clause), because is it based on the central word only. Same for P,N,G,PartSpeech} & Part-1.py \\ \hline
Anim & 1=animate, 2=non-animate & Part-1.py input \\ \hline
Human & 1=human (or divine), 2=non-human & Part-1.py input \\ \hline
MinGram & Minimal encoding possible in this grammatical context & \\ \hline
MinSem & Minimal encoding that would be semantically unambiguous in this context & \\ \hline
Ext & How extensive is the current reference? & \\ \hline
Nr & Counter how often the current referent (name) has been referred to in the corpus & Part-2.py \\ \hline
NewDSU & Is current reference a new participant, already mentioned, active participant in same role, or active participant in different role (all within DSU)?  & \\ \hline
NewChap & Is current reference a new participant, already mentioned, active participant in same role, or active participant in different role (all within chapter)?  & \\ \hline
Col & Is the participant an individual, collective or an individual in a compound design? & Part-1.py input \\ \hline
ColPart & Name of collective / compound / individuals to which current reference belongs\footnote{Overlap with ``Referents"} & Part-1.py input \\ \hline
MajDSU & Importance of the participant in current DSU & \\ \hline
MajChap & Importance of the participant in current chapter & \\ \hline
Var & Textual variants influencing the reading of this reference and its context & \\ \hline
Tag & Unique tag, composed of various place indicators: place (``a"), ClANr, PhANr, WNr, Nodetype & Part-1.py \\ \hline
Referents & Referents included in this reference (if not identical with Name, but relevant in textual context) & Part-1.py input \\ \hline
Cltype & Type of the clause to which reference belongs as given by F.etcbc4\_ft\_typ.v(node) & Part-1.py \\ \hline
Phtype & Type of the phrase to which reference belongs as given by F.etcbc4\_ft\_typ.v(node) & Part-1.py \\ \hline
Phfunc & Function of the phrase to which reference belongs as given by F.etcbc4\_ft\_function.v(node) & Part-1.py \\ \hline
SPhNr & Subphrase number & Part-1.py \\ \hline
Sub & Specifies participant type: independent or a kind of sub-participant (see workflow2.tex) & Part-1.py input \\ \hline
InfoStruc & About Information Structure (topic, focus, etc) & ?? \\ \hline
Center & Center of reference & ?? \\ \hline

%Enc\_opt & Encoding options given the cl\_type and ph\_func & Part-1.py \\ \hline
%Enc\_opt\_coded & Encoding options given the cl\_type and ph\_func, coded with numbers of prominence & Part-1.py \\ \hline
%Prom & Prominence of the encoding given the list enc\_opt & Part-1.py
\\ \hline

\section{gcdata.enc}
ID & Automatically created key & encoding_options.py \\ \hline
Cl_type & Clause_type (F.etcbc4_ft_typ.v(node)) & encoding_options.py \\ \hline
Ph_type & Phrase_type (F.etcbc4_ft_typ.v(node)) & \\ \hline
Ph_func & Phrase_function (F.etcbc4_ft_function.v(node)) & encoding_options.py \\ \hline
Enc_options & Options for encoding, given Cl_type and Ph_function & encoding_options.py \\ \hline
Enc_options_coded & Options for encoding, coded with numbers indicating prominence & encoding_options.py \\ \hline

\end{longtable}

\end{document}

